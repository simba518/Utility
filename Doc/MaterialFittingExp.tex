\documentclass[9pt,twocolumn]{extarticle}

\usepackage[hmargin=0.5in,tmargin=0.5in]{geometry}
\usepackage{amsmath,amssymb}
\usepackage{times}
\usepackage{graphicx}
\usepackage{subfigure}

\usepackage{cleveref}
\usepackage{color}
\newcommand{\TODO}[1]{\textcolor{red}{#1}}

\newcommand{\FPP}[2]{\frac{\partial #1}{\partial #2}}
\newcommand{\argmin}{\operatornamewithlimits{arg\ min}}
\author{Siwang Li}

\title{Full Space Material Fitting}

%% document begin here
\begin{document}
\maketitle

\setlength{\parskip}{0.5ex}

\section{Formulation}
Given the optimized modal basis $W$, the corresponding eigenvalues $\Lambda$,
and the mass matrix $\tilde{M}$, we solve for the Shear modulus
$G=(G_1,\cdots,G_{tet})$, Lame's first parameter $l=(l_1,\cdots,l_{tet})$ for
all tetrahedrons using this formulation
\begin{equation} \label{all}
  \min_{G,l}(\phi_e(G,l)+\mu_{g}\phi_s(G)+\mu_{l}\phi_s(l)), \mbox{s.t. } G_i>0, l_i>0
\end{equation}
where $\mu_g, \mu_l$ are some penalty parameters, and
\begin{equation} \label{smooth}
  \phi_s(x) = \frac{1}{2}\sum_{(i,j)_{n}}(x_i-x_j)^2V_{i,j}, \mbox{ for all
    tet. }i
\end{equation}
which indicates that, the material $x$ should be smooth, where $(i,j)_{n}$ represents a
neighbor tetrahedron pair, and $V_{i,j}=\frac{1}{4}(v_i+v_j)$ with $v_i$ is the
volume of tetrahedron $i$. 

The objective function $\phi_e(G,l)$ is defined as
\begin{equation} \label{diag_k}
  \phi_e(E) = \frac{1}{2}\|{W}^TK(G,l){W}-({W}^T\tilde{M}{W})\Lambda\|_F^2
\end{equation}
which is designed to enforce that the following motion equation can be
diagonalized basis matrix $W$,
\begin{equation} \label{motion_eq}
  \tilde{M}\ddot{x} + K(G,l)x = 0
\end{equation}

\paragraph{Scale lambda.}

\paragraph{Select modes.}

\paragraph{Constrained nodes.}
In all the experiments, the left end of the beam is fixed, and for these
constrained nodes, we first remove the corresponding columns and rows from
$\tilde{K}, \tilde{M}$, and remove the corresponding rows in $W$, then use
them to solve for $G$ and $l$.

\section{Experiments}
\subsection{Coarse mesh with smooth material}

\subsection{Coarse mesh with non-smooth material}

\subsection{Fine mesh with smooth material}

\subsection{Fine mesh with non-smooth material}

\section{Conclusion}


\end{document}