\documentclass[9pt,twocolumn]{extarticle}

\usepackage[hmargin=0.5in,tmargin=0.5in]{geometry}
\usepackage{amsmath,amssymb}
\usepackage{times}
\usepackage{graphicx}
\usepackage{subfigure}

\usepackage{cleveref}
\usepackage{color}
\newcommand{\TODO}[1]{\textcolor{red}{#1}}

\newcommand{\FPP}[2]{\frac{\partial #1}{\partial #2}}
\newcommand{\argmin}{\operatornamewithlimits{arg\ min}}
\author{Siwang Li}

\title{Full Space Material Fitting}

%% document begin here
\begin{document}
\maketitle

\setlength{\parskip}{0.5ex}

\section{Formulation}
Given the optimized modal basis $W$, the corresponding eigenvalues $\Lambda$,
and the mass matrix $\tilde{M}$, we solve for the Shear modulus
$G=(G_1,\cdots,G_{tet})$, Lame's first parameter $l=(l_1,\cdots,l_{tet})$ for
all tetrahedrons using this formulation
\begin{equation} \label{all}
  \min_{G,l}(\phi_e(G,l)+\mu_{g}\phi_s(G)+\mu_{l}\phi_s(l))
\end{equation}
\[
\mbox{subject to }  G_{max} \ge G_i\ge G_{min} \mbox{ and } l_{max} \ge l_i\ge l_{min}
\]
where $\mu_g, \mu_l$ are some penalty parameters, and
\begin{equation} \label{smooth}
  \phi_s(x) = \frac{1}{2}\sum_{(i,j)_{n}}(x_i-x_j)^2V_{i,j}, \mbox{ for all
    tet. }i
\end{equation}
which indicates that, the material $x$ should be smooth, where $(i,j)_{n}$ represents a
neighbor tetrahedron pair, and $V_{i,j}=\frac{1}{4}(v_i+v_j)$ with $v_i$ is the
volume of tetrahedron $i$. 

The objective function $\phi_e(G,l)$ is defined as
\begin{equation} \label{diag_k}
  \phi_e(G,l) = \frac{1}{2}\sum_{i=1}^{r}\mu_i^2\|{W}^T_iK(G,l){W}_i-({W}^T_i\tilde{M}{W}_i)\lambda_i\|_F^2
\end{equation}
% \begin{equation} \label{diag_k}
%   \phi_e(G,l) = \frac{1}{2}\|{W}^TK(G,l){W}-({W}^T\tilde{M}{W})\Lambda\|_F^2
% \end{equation}
which is designed to enforce that, for each mode $W_i$, we can convert the
following motion equation into the corresponding reduced motion equation by
using $W_i$, 
\begin{equation} \label{motion_eq}
  \tilde{M}\ddot{u} + K(G,l)u = 0
\end{equation}

I have also tried to define $\phi_e(G,l)$ as
\begin{equation} \label{dk}
  \phi_e(G,l)=\frac{1}{2}\|{W'}^TK(G,l){W'}-({W'}^T\tilde{M}{W'})\Lambda\|_F^2
\end{equation}
where each column of $W'$ is $W'_i = \mu_iW_i$. It works well for the coarse beam
model, but produce poor results for the fine beam model of our paper. I suppose
that, this is because there are some noise of these optimized mode, and the
requirements of minimizing
$\frac{1}{2}\|{W'}^T_iK(G,l){W'}_j-({W'}^T_i\tilde{M}{W'}_j)\lambda_j\|_F^2$ for
$i \neq j$ is too strict and no necessary to be enforced.

\paragraph{Scale $\Lambda$.}
Suppose that the optimized diagonal stiffness matrix is $\tilde{\Lambda}$, and
$h, \rho$ are the timestep and density for the reduced material optimization in
our paper, while $\rho_0$ and $h_0$ are the actual parameters for generating the
input animation, then we need to compute
\begin{equation} \label{scale_la}
  \Lambda = \frac{\rho_0 h^2}{\rho h_0^2}\tilde{\Lambda}
\end{equation}
For the beam model in our paper, we use $\rho_0=100, \rho=1, h_0=0.04$ and
$h=1$.

\paragraph{Penalty for each mode.}
The modes used to recover the full space material should have large
displacements and small control forces, and the modes with smaller frequency is
also much desirable than those with larger frequency. With these in
consideration, we compute the penalty $\mu_i$ for each mode using
\begin{equation} \label{mu-i}
  \mu_i = \frac{\|z_i\|_F^2}{\|f_i\|_F^2 \lambda_i^2}
\end{equation}
where $z_i, f_i$ are the optimized reduced displacements and control forces of
mode $i$ respectively.

\paragraph{Constrained nodes.}
In all the experiments, the left end of the beam is fixed, and for these
constrained nodes, we first remove the corresponding columns and rows from
$\tilde{K}, \tilde{M}$, and remove the corresponding rows in $W$, then use
them to solve for $G$ and $l$.

\begin{figure}[htb]
  \centering
  \newcommand{\Pic}[1]{
    \includegraphics[width=0.23\textwidth]{./figures/coarse#1.png}}
  \begin{tabular}{@{}cc@{}}
    \Pic{-real}&\Pic{1e-2}\\
    (a) ground truth& (b) $\mu_g=\mu_l=10^{-2}$\\
    \Pic{1e-4}&\Pic{1e-6}\\
    (c) $\mu_g=\mu_l=10^{-4}$& (d) $\mu_g=\mu_l=10^{-6}$\\
  \end{tabular}\vspace*{-1mm}
  \caption{Results for the coarse beam model with different penalties.}
  \label{fig_coarse}
\end{figure}

\section{Experiments and Analysis}
\subsection{Coarse mesh with non-smooth material}
In the first experiment, we use a coarse beam model with non-smooth materials ,
and the results are shown in figure \ref{fig_coarse}. In this experiment we can
found that
\begin{itemize}
\item When the penalties are properly selected ($10^{-4}$ for this experiment),
  the distribution of $E$ is much similar to the ground truth, while the
  difference in the Poisson's ratio is a little larger. Maybe we should fix
  Poisson's ratio, and optimize for $E$ only.
\item When we increase the penalties $\mu_g$ and $\mu_l$, the distribution of
  the material is more smooth, while the amplitude is become smaller. And this
  is the due to the smoother item adopted in equation (\ref{all}).
\item We also found that we need to provide some boundary for the solution of
  this problem. Here, we set the boundary of $G, l$ as $[1,10^{10}]$. If we
  simply set the lower boundary as $[0,+\infty]$, then the optimal function
  value of (\ref{all}) will be smaller, however, some of the results will be
  very small which is not desired.
\end{itemize}

\begin{figure}[htb]
  \centering
  \newcommand{\Pic}[1]{
    \includegraphics[width=0.48\textwidth]{./figures/fine#1.png}}
  \begin{tabular}{@{}c@{}}
    \Pic{-real}\\
    (a) ground truth\\
    \Pic{1e-4}\\
    (b) $\mu_g=\mu_l=10^{-4}$
  \end{tabular}\vspace*{-1mm}
  \caption{Results for the fine beam model.}
  \label{fig_fine}
\end{figure}

\subsection{Fine mesh with non-smooth material}
The results are shown in figure \ref{fig_fine}. With properly chose penalties
and boundaries, the distribution of $E, G$ is similar to the ground
truth. However, the magnitude is different and the distribution of $v, l$ are
also different. In this experiment, we found that, this result is much better
than use (\ref{dk}) as introduced in section 1.

\subsection{Fine mesh with smooth material}
TODO: The results analysis of this model will be provided soon.


\end{document}