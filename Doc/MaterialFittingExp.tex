\documentclass[9pt,twocolumn]{extarticle}

\usepackage[hmargin=0.5in,tmargin=0.5in]{geometry}
\usepackage{amsmath,amssymb}
\usepackage{times}
\usepackage{graphicx}
\usepackage{subfigure}

\usepackage{cleveref}
\usepackage{color}
\newcommand{\TODO}[1]{\textcolor{red}{#1}}

\newcommand{\FPP}[2]{\frac{\partial #1}{\partial #2}}
\newcommand{\argmin}{\operatornamewithlimits{arg\ min}}
\author{Siwang Li}

\title{Full Space Material Fitting}

%% document begin here
\begin{document}
\maketitle

\setlength{\parskip}{0.5ex}

\section{Formulation}
Our input include the optimized modal basis $W$, the corresponding eigenvalues
$\Lambda$, the mass matrix $\tilde{M}$, as well as the fixed Poisson's ratio
$v$, then we solve for the Young's parameters $E=(E_1,\cdots,E_{tet})$ for all
tetrahedrons using this formulation
\begin{equation} \label{all}
  \min_{E}(\phi_e(E)+\mu_{e}\phi_s(E)), \mbox{s.t. } E_i>0
\end{equation}
where $\mu_e$ are some penalty parameters, and
\begin{equation} \label{smooth}
  \phi_s(x) = \frac{1}{2}\sum_{(i,j)_{n}}(x_i-x_j)^2V_{i,j}, \mbox{ for all
    tet. }i
\end{equation}
which indicates that, the material $x$ should be smooth, where $(i,j)_{n}$ represents a
neighbor tetrahedron pair, and $V_{i,j}=\frac{1}{4}(v_i+v_j)$ with $v_i$ is the
volume of tetrahedron $i$. 

The objective function $\phi_e(E)$ is defined such that the following motion
equation can be diagonalized by some basis matrix $\tilde{W}$,
\begin{equation} \label{motion_eq}
  \tilde{M}\ddot{x} + K(E)x = 0
\end{equation}
where $\tilde{M}$ is the given mass matrix. To achieve this, we first solve for
the density $\rho$ using 
\begin{equation} \label{mass}
   \min_{\rho}\frac{1}{2}\|W^TM(\rho)W-I\|_F^2+\mu_{r}\phi_s(\rho), \mbox{s.t. }
   \rho_i>0
\end{equation}
Then transform the basis matrix $W$ as
\begin{equation} \label{W_ext}
  \tilde{W}=\tilde{M}^{-\frac{1}{2}}M^{\frac{1}{2}}W
\end{equation}
It is obviously that $\tilde{W}^T\tilde{M}\tilde{W}\approx I$, and we can solve
for $E$ by using
\begin{equation} \label{diag_k}
  \phi_e(E) = \frac{1}{2}\|\tilde{W}^TK(E)\tilde{W}-\Lambda\|_F^2
\end{equation}

\paragraph{Constrained nodes.}
In all the experiments, the left end of the beam is fixed, and for these
constrained nodes, we first remove the corresponding columns and rows from
$\tilde{K}, \tilde{M}, M$, and remove the corresponding rows in $W$, then use
them to solve for $E$.

\section{Experiments}



\end{document}