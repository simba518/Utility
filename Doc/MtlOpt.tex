\documentclass[9pt,twocolumn]{extarticle}

\usepackage[hmargin=0.5in,tmargin=0.5in]{geometry}
\usepackage{amsmath,amssymb}
\usepackage{times}
\usepackage{graphicx}
\usepackage{subfigure}

\usepackage{cleveref}
\usepackage{color}
\newcommand{\TODO}[1]{\textcolor{red}{#1}}

\newcommand{\FPP}[2]{\frac{\partial #1}{\partial #2}}
\newcommand{\argmin}{\operatornamewithlimits{arg\ min}}
\author{Siwang Li}

\title{Full Space Material Fitting}

%% document begin here
\begin{document}
\maketitle

\setlength{\parskip}{0.5ex}

\section{Summary} 
Given the modal basis $W$, eigenvalues $\Lambda$, and mass matrix ${M}$, we
propose an approach to recover the elastic material for all
tetrahedrons. Currently, we build our formulation based on two observations:
firstly, the motion in full space of the object with optimal material should be
reduced onto the subspace described by $W$ and $\Lambda$; Secondly, the material
should be smooth. Experiments demonstrate that
\begin{itemize}
\item The results significantly impacts by the penalty for the smooth item. When
  we choose a large enough penalty, we can usually obtain materials with desired
  distribution, but with much smaller magnitude. 
\item A penalty matrix for each mode can improve the result (see figure
  \ref{exp2_cmp_c}).
\item If we further fix the Poisson's ratio, and optimize for Young's parameter
  only, the results would be better (see figure \ref{exp1_smooth_GL} and figure
  \ref{exp1_smooth_E}).
\item For the fine model with a large number of tetrahedrons, it is costly to
  compute the hessian matrix, and we optimize the problem by using gradient
  only, and sometimes this method failed to convergent.
\item Currently, The best results we can obtained for the beam model with
  non-smooth material in our paper is given in figure
  \ref{exp2_opt_fine_ns}. There is two problems for this result: firstly, the
  resulting material is too soft; Secondly, distribution of the Poisson's ratio
  is far from the ground truth.
\end{itemize}

Several ways are worth to try to improve the results: firstly, use a Laplace
operator for the smooth item; Secondly, fix the Poisson's ratio, and optimize
for Young's parameter only (but why?); Thirdly, find a more robust numerical
method for solving the problem for large model; Finally, introduce additional
constraints to ensure that the major motion of the model with optimal material
is similar to the motion described by $W$ and $\Lambda$.

\begin{figure}[htb]
  \centering
  \newcommand{\PicCNS}[1]{
    \includegraphics[width=0.15\textwidth]{./figures/exp2_non_smooth_fine_GL_mu#1.png}}
  \begin{tabular}{@{}ccc@{}}
    \PicCNS{-4} & \PicCNS{1} & \PicCNS{50}\\
    \PicCNS{-4-v} & \PicCNS{1-v} & \PicCNS{50-v}\\
    (a) $\mu=10^{-4}$ & (b) $\mu=1$ & (c) $\mu=50$
  \end{tabular}\vspace*{-1mm}
  \caption{Results for the fine beam with non-smooth material used in our
    paper. Here, we set $x=(G,l)$.}
  \label{exp2_opt_fine_ns}
\end{figure}

\section{Formulation}
Given modal basis $W$, eigenvalues $\Lambda$, and mass matrix ${M}$, we solve
for the elastic material $x$ for all tetrahedrons using this formulation
\begin{equation} \label{all}
  \min_{x}(\phi_e(x)+\mu\phi_s(x))
\end{equation}
\[
\mbox{subject to }  x_{min} \le x_i\le x_{max}
\]
where $\mu$ are some penalty parameters, and
\begin{equation} \label{smooth}
  \phi_s(x) = \frac{1}{2}\sum_{(i,j)_{n}}(x_i-x_j)^2\tilde{V}_{i,j}, \mbox{ for all
    tet. }i
\end{equation}
which indicates that, the material $x$ should be smooth, where $(i,j)_{n}$
represents a neighbor tetrahedron pair, and
$\tilde{V}_{i,j}=\frac{1}{4}(V_i+V_j)$ with $V_i$ is the volume of tetrahedron
$i$. In our experiments, we define $x=(G,l)$ or $x=E$, where
$G=(G_1,\cdots,G_{tet})$, $l=(l_1,\cdots,l_{tet})$ are the Shear modulus, Lame's
first parameter and Young's parameter for each tetrahedron respectively. When we
define $x=E$, we also assume that the Poisson's ratio $v$ are given.

Suppose $z(t)$ is the solution of the following reduced motion equation
\begin{equation} \label{red_meq}
  \ddot{z} + \Lambda z = 0
\end{equation}
for given $\Lambda, W$. Then for the optimal material, we require $z(t)$ to
satisfy
\begin{equation} \label{full_meq_red}
  (W^TMW)\ddot{z} + (W^TK(x)W) z = 0
\end{equation}
If $\det(W^TMW) > 0$, then this requirement is satisfied if and only if
\begin{equation} \label{cond_1}
  W^TK(x)W = (W^TMW) \Lambda
\end{equation}
Thus we define 
\begin{eqnarray}\label{dk}
  \phi_e(x) &=& \frac{1}{2}\|{W}^TK(x){W}-({W}^T{M}{W})\Lambda\|_C^2\\
  &=&\frac{1}{2}\sum_{i=1}^{r}\sum_{j=1}^{r}C_{i,j}^2\|{W}^T_iK(x){W}_j-({W}^T_i{M}{W}_j)\lambda_j\|_F^2
  \nonumber
\end{eqnarray}
Here, $C$ is a penalty matrix for each mode.

\paragraph{Scale $\Lambda$.}
In our experiments, we assume that the density is uniform, while $\rho$ and
$\rho_0$ are the density for the reduced material optimization and the desired
density for material recovering respectively. Suppose that $h$ is the time step
for reduced material optimization, $h_0$ is the time step for generating the
input animation, while the optimized diagonal stiffness matrix is
$\tilde{\Lambda}$, then we compute
\begin{equation} \label{scale_la}
  \Lambda = \frac{\rho_0 h^2}{\rho h_0^2}\tilde{\Lambda}
\end{equation}
For the beam model in our paper, we use $\rho_0=100, \rho=1, h_0=0.04$ and
$h=1$. 

\paragraph{Penalty for each mode.}
The modes used to recover the full space material should have large
displacements and small control forces, and the modes with smaller frequency is
also much desirable than those with larger frequency. With these in
consideration, we compute the penalty $\mu_i$ for each mode using
\begin{equation} \label{mu-i}
  \mu_i = \frac{\|z_i\|_F^2}{\|f_i\|_F^2 \lambda_i^2}
\end{equation}
Then we define
\begin{equation} \label{mu-i}
  C_{i,j} = \min(\mu_i, \mu_j)
\end{equation}
where $z_i, f_i$ are the optimized reduced displacements and control forces of
mode $i$ respectively.

\paragraph{Constrained nodes.}
In all the experiments, the left end of the beam is fixed, and for these
constrained nodes, we first remove the corresponding columns and rows from
$\tilde{K}, \tilde{M}$, and remove the corresponding rows in $W$, then use
them to solve for $G$ and $l$.

\begin{figure}[htb]
  \centering
  \newcommand{\Pic}[1]{
    \includegraphics[width=0.14\textwidth]{./figures/exp1_smooth_#1_E.png}}
  \begin{tabular}{@{}ccc@{}}
    \Pic{real} &
    \Pic{mu0} &
    \Pic{mu-4} \\
    (a) ground truth &
    (b) $\mu=10^{-4}$.&
    (c) $\mu=0$.
  \end{tabular}\vspace*{-1mm}
  \caption{Results for the coarse beam model with smooth material, and optimize
    for $x=(G,l)$.}
  \label{exp1_smooth_GL}
\end{figure}

\begin{figure}[htb]
  \centering
  \newcommand{\Pic}[1]{
    \includegraphics[width=0.24\textwidth]{./figures/exp1_smooth_#1_E.png}}
  \begin{tabular}{@{}cc@{}}
    \Pic{E_mu0} & \Pic{E_mu-4}\\
    (a) $\mu=0$ & (b) $\mu=10^{-4}$
  \end{tabular}\vspace*{-1mm}
  \caption{Results for the coarse beam model with smooth material, and fix
    $v=0.45$, optimize for $x=E$.}
  \label{exp1_smooth_E}
\end{figure}

\begin{figure}[htb]
  \centering
  \newcommand{\Pic}[1]{
    \includegraphics[width=0.14\textwidth]{./figures/exp1_non_smooth_#1.png}}
  \begin{tabular}{@{}ccc@{}}
    \Pic{real} & \Pic{GL_mu-4} & \Pic{E_mu-4}\\
    (a) ground truth & (b) $x=(G,l)$ & (c) $x=E$
  \end{tabular}\vspace*{-1mm}
  \caption{Results for the coarse beam model with non-smooth material. We use
    $\mu=10^{-4}$ to produce these results.}
  \label{exp1_non_smooth_E}
\end{figure}

\section{Experiments and Analysis}
\subsection{Experiment 1}
We generate $W$ and $\Lambda$ using Modal Analysis for different models with
non-uniform Young's parameter $E_0$ and uniform Poisson's ratio $v_0$, and
density $\rho_0$, then solve for eq. (\ref{all}) using IPOPT to recover the
elastic material. In these experiments, we set $C_{i,j}=1$. Following are the
details.

\begin{itemize}
\item Figure \ref{exp1_smooth_GL} and figure \ref{exp1_smooth_E} are the results
  of a coarse beam model with smooth material. It can be seen that, optimize for
  $x=E$ produce better results than using $x=(G,l)$. We also notice that, the
  hessian matrix $\frac{\partial^2{\phi_e}}{\partial^2{x}}$ is ill conditioning,
  and if we don't use hessian information in the optimization, the optimization
  convergent very slow even use the ground truth material as the initial value.

\item The results of a coarse beam model with non-smooth material is shown in
  figure \ref{exp1_non_smooth_E}, which demonstrate that this method can produce
  results that are similar to the ground truth for object with non-smooth
  material. However, the magnitude of the resulting material are significantly
  different from the ground truth material. In particular, the material of some
  tetrahedrons are very soft. Here, we set $x_{min}=0.01$ as the lower bound in
  eq. (\ref{all}), and if we set smaller $x_{min}$, the smallest material in
  figure \ref{exp1_non_smooth_E} will also be much smaller.

\item The results for a fine beam model with smooth and non-smooth material are
  shown in figure \ref{exp1_smooth_fine}. As the computation of
  $\frac{\partial^2{\phi_e}}{\partial^2{x}}$ too costly, we use IPOPT to
  optimize it without using hessian information, and use the ground truth
  material as the initial value. When $\mu=10^{-6}$, the solution is not
  convergent. For $\mu=10^{2}$, the solution is convergent, but the results are
  significantly different from the ground truth. Thus, for large model, we need
  to find a more robust numerical approach.
\end{itemize}

\begin{figure}[htb]
  \centering
  \newcommand{\Pic}[1]{
    \includegraphics[width=0.14\textwidth]{./figures/exp1_smooth_fine_#1.png}}
  \newcommand{\PicNS}[1]{
    \includegraphics[width=0.14\textwidth]{./figures/exp1_non_smooth_fine_#1.png}}
  \begin{tabular}{@{}ccc@{}}
    \Pic{real} & \Pic{E_mu-6} & \Pic{E_mu100}\\
    (a) ground truth & (b) $\mu=10^{-6}$ & (c) $\mu=10^{2}$\\
    \PicNS{real} & \PicNS{GL_mu-6} & \PicNS{GL_mu100}\\
    (d) ground truth & (e) $\mu=10^{-6}$ & (f) $\mu=10^{2}$
  \end{tabular}\vspace*{-1mm}
  \caption{Results for the fine beam model with smooth (top row) and
    non-smooth (bottom row) material, where we fix $v=v_0$, and optimize for
    $x=E$.}
  \label{exp1_smooth_fine}
\end{figure}

\begin{figure}[htb]
  \centering
  \newcommand{\PicCNS}[1]{
    \includegraphics[width=0.14\textwidth]{./figures/exp2_non_smooth_E_mu#1.png}}
  \newcommand{\PicFNS}[1]{
    \includegraphics[width=0.14\textwidth]{./figures/exp2_non_smooth_E_mu#1.png}}
  \newcommand{\PicF}[1]{
    \includegraphics[width=0.14\textwidth]{./figures/exp2_smooth_fine_E_mu#1.png}}
  \begin{tabular}{@{}ccc@{}}
    \PicCNS{-8} & \PicCNS{-2} & \PicCNS{100}\\
    (a) $\mu=10^{-8}$ & (b) $\mu=10^{-2}$ & (c) $\mu=10^{2}$\\
    \PicF{-10} & \PicF{-6} & \PicF{-2}\\
    (d) $\mu=10^{-10}$ & (e) $\mu=10^{-6}$ & (f) $\mu=10^{-2}$
  \end{tabular}\vspace*{-1mm}
  \caption{Results for the different beam model using the optimized $W$,
    $\Lambda$ as input. From top to bottom row: results for the coarse model
    with non-smooth material and fine model with smooth material. The ground
    truth material for these model can be found in previous figures.}
  \label{exp2_opt}
\end{figure}

\begin{figure}[htb]
  \centering
  \newcommand{\PicCNS}[1]{
    \includegraphics[width=0.2\textwidth]{./figures/exp2_non_smooth_E_mu#1.png}}
  \begin{tabular}{@{}cc@{}}
    \PicCNS{-2} & \PicCNS{-2-c1}\\
    (a) using mode penalty & (b) without mode penalty
  \end{tabular}\vspace*{-1mm}
  \caption{Comparison of the results with and without using mode penalty matrix
    $C$. In both experiments we use $\mu=10^{-2}$ and set $x=E$.}
  \label{exp2_cmp_c}
\end{figure}

\subsection{Experiment 2}
For $W$ and $\Lambda$ obtained from reduced material optimization, solve
eq. (\ref{all}) to obtain the results. For the coarse model we use the hessian
information, while for the fine model we do not. The ground truth material for
these experiments can be found in figure \ref{exp1_non_smooth_E}(a),
\ref{exp1_smooth_fine}(a) and \ref{exp1_smooth_fine}(d).

\begin{itemize}
\item We check the results for different models using different penalty $\mu$
  (see figure \ref{exp2_opt}). We fix Poisson's ratio $v$, and optimize for
  $x=E$. The penalty $\mu$ significantly impacts the results. When we increase
  $\mu$, the magnitude of the material will be smaller, and a Laplace method can
  be used to avoid such problem. For the fine beam model with smooth material,
  we can obtain desired results by using $\mu=10^{-6}$.
\item Figure \ref{exp2_cmp_c} show that, the mode penalty matrix $C$ can improve
  the results. We have also compared the results with other penalty $\mu$, and
  observed the similar results. 
\item Finally, we give the results for the fine beam model with non-smooth
  material used in our paper in figure \ref{exp2_opt_fine_ns}. Here, we optimize
  for $x=(G,l)$, and convert $E=1$, $v=0.45$ to $G$, $l$ as the initial
  value. We can obtain similar material distribution for large $\mu$, however
  with much smaller magnitude. Another observation is that, the distribution of
  the Poisson's ration $v$ is opposite to the Young's parameter $E$.
\end{itemize}

\end{document}